\documentclass[11pt,letterpaper,sans,english]{moderncv}        % possible options include font size ('10pt', '11pt' and '12pt'), paper size ('a4paper', 'letterpaper', 'a5paper', 'legalpaper', 'executivepaper' and 'landscape') and font family ('sans' and 'roman')

\moderncvstyle{classic}                             % style options are 'casual' (default), 'classic', 'oldstyle' and 'banking'
\moderncvcolor{black}                               % color options 'blue' (default), 'orange', 'green', 'red', 'purple', 'grey' and 'black'
%\nopagenumbers{}                                  % uncomment to suppress automatic page numbering for CVs longer than one page
\usepackage[utf8]{inputenc}                       % if you are not using xelatex ou lualatex, replace by the encoding you are using
\usepackage[scale= .85, letterpaper]{geometry}
\usepackage{babel}
\usepackage{tabu}



%----------------------------------------------------------------------------------
%            personal data
%----------------------------------------------------------------------------------

\firstname{Huyanh}
\familyname{Hoang}
%\title{Resumé title}                               % optional, remove/comment the line if not wanted

\address{115 Lehigh Aisle, Irvine, CA 92612}{}{}         % optional, remove/comment the line if not wanted; the "country" arguments can be omitted or provided empty
\mobile{(562)-666-1609}                          % optional, remove/comment the line if not wanted
%phone{phone number}                           % optional, remove/comment the line if not wanted
%\fax{fax number}                             % optional, remove/comment the line if not wanted
\email{huyanhh@uci.edu}                               % optional, remove/comment the line if not wanted
\homepage{huyanhhoang.me}                         % optional, remove/comment the line if not wanted
\social[linkedin][www.linkedin.com/in/huyanhh]{huyanhh}
\social[github][github.com/huyanhh]{huyanhh}

% optional, remove/comment the line if not wanted
% \photo[64pt][0.4pt]{picture}                       % optional, uncomment the line if wanted; '64pt' is the height the picture must be resized to, 0.4pt is the thickness of the frame around it (put it to 0pt for no frame) and 'picture' is the name of the picture file
%\quote{"Failure is an option here. If things are not failing, you are not innovating enough.”}                                 % optional, remove/comment the line if not wanted
%
\begin{document}
%-----       resume       ---------------------------------------------------------
\makecvtitle
\section{Education}
\cventry{2013--June 2017}{Bachelor of Science, Informatics}{University of California, Irvine}{}{}{Cumulative GPA: 3.4}  % arguments 3 to 6 can be left empty
\section{Professional Experience}
\cventry{January 2016--Present}{iOS Development Intern}{Rohm}{Irvine}{}{\begin{itemize}
		\item Converted prototypes for an iOS platform dealing with event coordination and event discovery to Swift code using external libraries and UIKit
		\item Overhauled existing modules to fix programming bugs created from outsourced code
		\item Constructed documentation detailing a high-level overview of the code structure for later implementation in the Android version
	\end{itemize}}

\section{Software Projects}
%cventry{year--year}{degree or job title}{institution or employer}{city}{grade}{description}}
\cventry{April 2016--Present}{paso}{LAHacks 2016}{Los Angeles}{}{\begin{itemize}
		\item Collaborated with non iOS developers to extract requirements to develop a fitness iOS application that donates a set amount of money to charity once the user walks 10,000 steps
		\item Adapted OAuth2 libraries to connect the backend FitBit API to the frontend logic
		\item Guided team members on UIKit, Swift, and UX design to make the application more user-centric
	\end{itemize}}
\cventry{April 2016}{FlagTraveler}{2016 NASA Space Apps Challenge}{Irvine}{}{\begin{itemize}
		\item Designed a solution to help users learn new facts about NASA's historical space events that would otherwise be undiscovered
		\item Engineered an iOS application that uses an HTML parser to scrape data sets from NASA's archives and displays them in a geotagging based interface
		\item Presented a pitch to the local community and received showcasing on open.NASA's Innovation Space
	\end{itemize}}

%\cventry{February 2016}{Procrastinationation}{TreeHacks}{Stanford}{}{\begin{itemize}
%		\item Collaborated with a multi-disciplinary team to create a Google Chrome extension that keeps track of active tabs pages of browser activity and visually displays that information using the highcharts framework
%		\item Led the application design process to mind-map the most important features
%		\item Designed the front-end features by writing scripts in HTML/CSS using the Bootstrap 3 framework
%	\end{itemize}}
\cventry{November 2015}{Unavoidable}{HackUCI}{Irvine}{}{\begin{itemize}
			\item Developed a first-person survival horror game for the Oculus Rift Development Kit 2
			\item Designed the map layout, menu scene, and gameplay in which the player must escape a labyrinth while being chased by an enemy AI
			\item Learned the basics of the Unity3D 5 Engine and C\# scripting to apply them all within a 24 hour timeframe
		\end{itemize}}
%\cventry{November 2015}{Medthodical}{MedAppJam}{Irvine}{}{\begin{itemize}
%		\item Participated in a competition with a multi-disciplinary team to create a medical iOS app in which physicians can send notifications to their patients indicating an appointment delay
%		\item Led the software engineering process by creating mockups using Sketch and Balsamiq, building a requirements document, and then writing code in Swift using the Xcode IDE
%	\end{itemize}}

%\cventry{March 2015}{Connect Four, Othello}{}{}{}{\begin{itemize}
%		\item Implemented a console-based implementation of Connect Four and Othello written in Python
%		\item Learned introductory socket and protocol concepts and used them to play Connect Four with a server AI
%		\item Learned to build a graphical user interface for the Othello game logic using tkinter
%	\end{itemize}}

%\section{Languages}
%\cvitemwithcomment{Language 1}{Skill level}{Comment}
%\cvitemwithcomment{Language 2}{Skill level}{Comment}
%\cvitemwithcomment{Language 3}{Skill level}{Comment}
%\section{Computer skills}
%\cvdoubleitem{Category 1}{Comment}{Category 4}{Comment}
%\cvdoubleitem{Category 2}{Comment}{Category 5}{Comment}
%\cvdoubleitem{Category 3}{Comment}{Category 6}{Comment}
%\section{Interests}
%\cvitem{Hobby 1}{Description}
%\cvitem{Hobby 2}{Description}
%\cvitem{Hobby 3}{Description}
\section{Leadership and Community}
\cventry{June 2013}{Eagle Scout Project}{Boy Scouts of America}{Westminster}{}{\begin{itemize}
		\item Developed and led a project involving the creation of wooden storage shelves for children at a local elementary school
		\item Managed a group of 15 people and assigned them different tasks such as painting and nailing
	\end{itemize}}
\section{Skills}
\cvlistitem{Proficiency in object-oriented programming with Python, Java, C++ and Swift, SQL}
\cvlistitem{Familiar with Xcode/Cocoa, Unified Modeling Language, HTML5/CSS3, Sketch, D3, MongoDB, Node.js}
%\section{References}
%\begin{cvcolumns}
%  \cvcolumn{Category 1}{Comment}
%  \cvcolumn{Category 2}{Comment}
%  \cvcolumn{Category 3}{Comment}
%\end{cvcolumns}
\clearpage
%-----       letter       ---------------------------------------------------------
% recipient data
%\recipient{Procore Recruitment Team}{Procore\\6309 Carpinteria Avenue\\Carpinteria, CA 93013}
%\date{\today}
%\opening{Dear Sir or Madam,}
%\closing{Yours faithfully,}
%%\enclosure{enclosures}          % use an optional argument to use a string other than "Enclosure", or redefine \enclname
%\makelettertitle
%I am interested in an iOS development internship at Procore because I was recommended by Daniel Phillips at the UCI Winter career fair. I seek to make nontrivial applications with my iOS skills and I believe that at Procore with its open, collaborative culture, I will be able to further develop my skills.\\[3mm]
%I am particularly very fascinated by the field of product management and its integration with backgrounds concerning user experience, business, and technology. I have been working with iOS for around a month now, and I am currently involved in two side projects that are helping me increase my proficiency. By the end of these projects, I will have attained a good amount of practical experience. What I am really needing, however, is a chance to work with more experienced mentors in an agile or lean environment. Through projects such as my Eagle Project and HackUCI, I understand the importance of establishing team cooperation by having a shared passion and a mutual vision of accomplishing set goals. By working with product managers and design engineers, I will be able to have a sense of the key heuristics that it takes to create a successful product.
%\\[3mm]
%I feel that with my strong motivation and passion in iOS and product management, I will be a great fit with the environment at Procore. I would like the opportunity to have an interview in order to discuss my experiences in more detail. I am best reached through email, and I will follow up with you in two weeks to check the status of my application. Thank you for your consideration.
%%first paragraph talk about what stands out, in the second detail the projects and move what i want to the third paragraph
%
%\makeletterclosing
%

\end{document}
